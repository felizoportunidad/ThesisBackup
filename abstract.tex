\prefacesection{Abstract}

Treatment planning is an essential step in radiation therapy. Clinical planning is resource limited - the number of radiation oncologists and the amount of time allocated on each plan are difficult to satisfy the increasing number of cancer patients. Current clinical planning is hampered due of the amount of time planning takes. 

One of the most time consuming aspect of planning is the dose calculation. Oversimplified models which are fast lead to inaccuracies and results that are not clinically acceptable. In this dissertation, we address how to speed up dose calculation using state-of-the-art cloud computing technology. The approach is evaluated for pencil beams and broad beams of high-energy electrons and photons. The cloud-based Monte Carlo simulation is compared to single-threaded implementation and demonstrates significant speed up. 

Current clinical treatment planning requires multiple trial-and-error adjustments of system parameters, which is costly and labor intensive. In such situation, current planners need to manually adjust system parameters.  In this dissertation, an autonomous treatment planning technique is implemented in a clinical platform. An outer-loop decision function is implemented as a feedback mechanism that interacts on-the-fly with a clinical platform.  The approach is applied to plan a head and neck VMAT case and a prostate IMRT case. The proposed technique shows effective autopiloting ability as well as promising practical usage for existing clinical systems.

In this dissertation, a strategy of using population-based prior knowledge for autopiloting treatment planning is implemented successfully in the platform of a commercial TPS. The dosimetric characteristics of a group of reference treatment plans are assembled and used to guide the auto-adjustment of the TPS parameters and the search for a clinically sensible plan. It is shown that a third-party TPS can be autopiloted effectively with incorporation of population-based knowledge of treatment plans.
