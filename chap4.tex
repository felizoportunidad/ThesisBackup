\chapter{Toward Big-Data Driven Treatment Planning: VMAT/IMRT Inverse Planning Autopiloted by Population-Based Prior Data}  \label{cha:chap4}
        
 \section{Introduction}
Inverse planning derives a patient-specific treatment plan through iterative interactions with an objective function, whose role is to mathematically rank a candidate treatment plan.  While the approach has led to clinical implementation of IMRT and VMAT, the planning process routinely used in the clinics is rather tedious and labor intensive. The underlying issue responsible for this problem is the involvement of multiple model parameters, i.e. the weighting factors and prescription in the objective function, in treatment planning \cite{xing1999a, yu2000}. Ideally, these parameters should be optimized before or together with the machine parameters. In practice, these parameters are determined through manual trial-and-errors since their influence on the final dose distribution is not known until an optimization with a given set of parameters is done. In the past two decades, many attempts have been made to improve the manual selection process of the model parameters so that the planning can be done more efficiently. It was demonstrated, for example, the determination of the weighting factors can be automated along with the fluence optimization by algorithmically mimicking a human planner's planning process with effective use of prior knowledge\cite{good2013,lian2013,purdie2011,xing1999a,zaghian2014}. While this strategy works for structure-specific weighting, specific formulation of population-based prior knowledge has not been well studied and effectively utilized to facilitate the process. 

In this work we develop a strategy of using population-based prior knowledge to autopilot the IMRT and VMAT inverse planning process. We use a commercial treatment planning system (TPS) as the backbone of treatment planning and dose calculation. The manipulation of the ``black-box'' TPS is technically realized by using an in-house software tool through the use of a set of plug-in subroutines. Algorithmically, the automation is achieved by iterative maneuver of TPS parameters under the guidance of an assemble of prior treatment plans (or a priori knowledge). Instead of using a single reference plan, which may either over- or under-constrain the decision-making process and compromise the final solution as a perfect reference that fits snugly with the case under planning is hard to find, here we propose to use an assemble of reference plans to construct a statistical decision-function to guide the solution search.  In doing so, the approach mitigates the potential problem of ``being stuck'' with a single non-optimal reference plan, thus enables us to better balance various conflicting dosimetric demands during treatment planning and obtain clinically sensible plans.
 
\section{Methods}
 Our treatment planning system consisted of three essential components: (i) a conventional TPS to find a "optimal" solution for a given set of model parameters; (ii) a robust formulation of prior knowledge for guiding the plan selection process; (iii) a decision-making (or outer-loop optimization) algorithm that evaluates plans with different tradeoff strategies and navigate the search for a solution consistent with the prior knowledge. The following summarizes the details of the technique. 
 
 \subsection{Software platform for autopiloted planning}
The workflow for the population knowledge-based autonomous planning is plotted in Fig. \ref{fig4.1}. An independent C$\sharp$ program is written to accomplish the tasks delineated in Fig. \ref{fig4.1}. In this program, the interactions with a commercial Eclipse TPS (Varian Medical Systems, Palo Alto, CA) are realized through a series of subroutines that are pre-recorded actions of a planner in operating the TPS by using Microsoft Visual Studio Coded UI\cite{johnson2015,website2015}. For example, to perform a 3D dose calculation in Eclipse, ``F5'' is clicked after the beam setup is done. A subroutine for performing dose calculation can be generated by recording this action, which can be called when dose distribution needs to be updated in the C$\sharp$ program. Coded UI provides a unique framework for us to probe and manipulate UI elements of TPS in application programming in C$\sharp$ or other programming environment. We have recorded all essential actions in Eclipse operation and saved them into a library of subroutines to query the black-box TPS through the system output for specified values of system inputs, to assess the quality of a plan generated by the TPS, and to provide updated model parameters to refine the Eclipse plan.

\begin{figure}
	\centering
	\includegraphics[width=12cm]{./figures/chap4/fig1.pdf}
	\caption{A flowchart of the autopiloted plan optimization scheme. An outer-loop calculation (indicated by the dashed rectangular box) analyzes the generated Eclipse plan and feeds the Eclipse optimizer with updated parameters for iterative improvement of the treatment plan. The gray boxes are realized through the use of recorded Eclipse actions. 
	\label{fig4.1}}
    \end{figure}

\subsection{Autopiloted VMAT/IMRT treatment planning}
\textit{Library of reference cases}: For a given patient, a set of reference plans with similar anatomy is chosen automatically with some pre-defined geometric criteria. Specifically, the images of current case are overlaid with a candidate reference plan from a library of previously treated patients. For each structure, the signed difference of the contour points of the current and reference plans is computed for ray lines starting from the center of the mass of the structure\cite{schreibmann2005}. The points for a ray-line to be in and out of the structure are recorded. The signed difference of an intercepting point is given by subtracting the radial distance of the point in the current case from that of the reference case. A plan is not considered as a good reference if the signed difference of any intercepting point in any structure (including the skin contour) is greater than 1-3 mm, depending on the size of the structure. 

\textit{Autopilot process}: The autopiloted planning involves the following key steps: (1) obtaining a candidate Eclipse plan, (2) comparing the Eclipse plan with reference data, (3) deriving a new set of Eclipse planning parameters, and (4) updating the Eclipse planning parameters and obtaining a new Eclipse plan. Formally, the plan selection process is to solve the following are: 

 \begin{align}\label{eq:eq4.1}
& \text{opt} \quad \text{TPS objective function with constraints} \nonumber\\
& \text{s.t.} \quad C^{ref_{l}}_{\sigma j} < C_{\sigma j} < C^{ref_{h}}_{\sigma j} \\
& \quad \quad (\sigma \in 1,2,...,N; j \in 1,2,...,J) \nonumber
\end{align}

where the optimization of TPS objective function with constraints is done by the Eclipse TPS $C_{\sigma j}$,  $C^{ref_{l}}_{\sigma j}$, and $C^{ref_{h}}_{\sigma j}$ are the dosimetric characteristic variable  of the j-th of structure $\sigma$ for the current and reference plans, respectively. The values of $C^{ref_{l}}_{\sigma j}$ and $C^{ref_{h}}_{\sigma j}$ are extracted from the reference plan library and define the allowed deviation of the j-th DVH segment of the structure $\sigma$. Note that here the prior knowledge sets our preferred variation range of the dosmetric quantity $C_{\sigma j}$. In order to speed up the calculation, the beam and weighting parameters of a plan in the middle of the assemble reference plans are used to ``warm-start'' of the autopiloted planning.  During the calculation, instead of letting the $C_{\sigma j}$  stop anywhere when its value is inside the range defined by $C^{ref_{l}}_{\sigma j}$ and $C^{ref_{h}}_{\sigma j}$ , we prioritize  the  $C^{ref}_{\sigma j}$ value by a preference function, which is a density function distribution derived from the frequency that a $C_{\sigma j}$  occurs. For a DVH segment of an organ at risk (OAR), we heuristically set the highest and lowest preference levels at the lower and higher ends of the range, respectively.  Similar is done for the PTV. Using an assemble of reference plans allows us to minimizing any potential deteriorating effect caused by the anatomical deviations of the reference plans. During the autopilot process, we examine the values of both $\{C_{\sigma j}\}$ and the corresponding preference levels at each iteration. An adjustment is made in those DVH segments which have the least preference values at the end of each iterative calculation step. The goal of our autopilot process here is to find a solution that is most consistent with the variation range and  preference levels derived based on the library of reference plans for all the relevant $C_{\sigma j}$.  At the end of iterative calculation, we let each resultant DVH segment to make a trial movement toward better PTV coverage or OAR sparing even if the segment has reached the highest preference level set by the reference plan. The trial movement is accepted if the plan is improved. This would allow continuous improvement of the solution when there is a room for improvement.

\subsection{Evaluation}
The above technique is applied to plan two clinical cases:  a five-field IMRT prostate case and a VMAT head and neck (HN) case. In the prostate IMRT case, 6 MV photon energy is used and the beam angles are 0o, 50o, 100o, 260o, and 310o, respectively. 78 Gy is prescribed to cover V95 of the PTV in 39 fractions.  For comparison, the resultant dose distributions of the autopilot scheme are compared with the corresponding plans used for clinical treatments. To construct a density function distribution, 15 previously treated prostate cancer patients are selected using the procedure outlined in Sec. 2.B.  For the HN case, two 360o 6 MV arcs arc used.  9 previously treated HN cases are selected as reference plans.

\section{Results}

\subsection{Five-field IMRT prostate treatment}
In Fig. \ref{fig4.2}, we show the band of DVH representing the preferred range and distribution of the resultant DVH curve for the prostate case under planning To illustrate the progressive improvement of the autopilot process, in Fig. \ref{fig4.3} we show the DVHs results of involved structures for the 1st, 7th, and 14th iteration. The improvement saturate after about 14 iterations and the calculation thus terminates after the 14th iteration. Fig. \ref{fig4.4} shows the DVH comparison between the clinical and autopiloted plans for the case. Fig. \ref{fig4.5} shows the isodose distribution of the two plans. Only the minor discrepancy is seen between the autopiloted plan obtained under the guidance of the population-based library of reference plans and clinical plan generated by a human planner independently. 

\begin{figure}
	\centering
	\includegraphics[width=12cm]{./figures/chap4/fig2.pdf}
	\caption{Plots of reference DVHs of  bladder (a), rectum (b), and PTV (c) for the prostate case under planning. The assemble of DVHs represents the preferred range of the resultant DVH curve. 
	\label{fig4.2}}
    \end{figure}

\begin{figure}
	\centering
	\includegraphics[width=12cm]{./figures/chap4/fig3.pdf}
	\caption{DVHs of the bladder, rectum, and prostate at iteration \#1 (dotted), \#7 (dashed), and\#14 (solid).  A systematic improvement in the DVHs is observed. 
	\label{fig4.3}}
    \end{figure}

\begin{figure}
	\centering
	\includegraphics[width=12cm]{./figures/chap4/fig4.pdf}
	\caption{A comparison of DVHs of the clinical and autopiloted plans for the five-field IMRT prostate case. The dashed and solid curves represent the DVHs of clinical and autopiloted plans, respectively. 
	\label{fig4.4}}
    \end{figure}

\begin{figure}
	\centering
	\includegraphics[width=12cm]{./figures/chap4/fig5.pdf}
	\caption{Side-by-side comparison of the isodose distributions of autopiloted (right) and clinical (left) plans for the prostate case. 
	\label{fig4.5}}
    \end{figure}

\subsection{HN VMAT case}
In Fig. \ref{fig4.6}, we show the assemble of DVHs of the spinal cord, brainstem and PTV of the reference plans obtained using the method of Sec. 2.B for the HN case under planning. Figure \ref{fig4.7} shows the DVH comparison between the clinical and autopiloted planning for the case, and Fig. \ref{fig4.8} shows the isodose distributions of the two plans.

\begin{figure}
	\centering
	\includegraphics[width=12cm]{./figures/chap4/fig6.pdf}
	\caption{Plots of reference DVH curves of spinal cord (a), brainstem (b), and PTV (c) for the head and neck case under planning. The assemble of DVHs presents the preferred range of the resultant DVH curve. 
	\label{fig4.6}}
    \end{figure}
   
\begin{figure}
	\centering
	\includegraphics[width=12cm]{./figures/chap4/fig7.pdf}
	\caption{A comparison of DVHs of the clinical and autopiloted plans for the two-arc head and neck VMAT case. The dashed and solid curves represent the DVHs of clinical and autopiloted plans, respectively. 
	\label{fig4.7}}
    \end{figure}
    
\begin{figure}
	\centering
	\includegraphics[width=12cm]{./figures/chap4/fig8.pdf}
	\caption{Side-by-side comparison of the isodose distributions of autopiloted (right) and clinical (left) plans for the head and neck case. 
	\label{fig4.8}}
    \end{figure}
     
\section{Discussion}
In inverse planning, an optimized solution is obtained under the guidance of an objective function containing multiple model parameters\cite{amit2015,chan2006,good2013,kim2015,li2000,wu2013,xing1999a, zhang2014}. In the past two decades, much of the efforts in research and commercial product development have been focused on finding a ``super powerful'' function with mathematical constraints to provide clinically sensible solution. While the formulation of objective function and the technique used to search through the solution space under the guidance of the objective function are important to the success of inverse planning, clinical IMRT/VMAT planning has been handicapped by the involvement of multiple model parameters that necessitates trial and error determination. In this paper, we extended our previous works on an estimation theory-based inverse planning and weighting factor optimization to autopilot the VMAT/IMRT planning process. The automation is realized by combining the functionalities of the black-box TPS and a decision-function constructed with incorporation of population-based prior knowledge. As illustrated in Fig. \ref{fig4.1}, the calculation proceeds is analogous to the planning process of a human planner, with a candidate plan assessed iteratively by the decision-function each time after the Eclipse optimization is done\cite{good2013,xing1999a}. During the autopiloted planning, the C$\sharp$ program interacts with TPS continuously to extract the updated information and to instruct on what to do next until a satisfactory plan is obtained. 

A salient feature of the work here is that the functionalities of a commercial TPS are utilized effectively, by recording the mouse clicks/keystrokes for specific tasks during planning as executable subroutines using the Coded UI. The recorded subroutines are called back in C$\sharp$ application grogram to accomplish a designated task. Instead of attempting to improve the TPS optimization algorithm, which is typically out of the control of a TPS user, the programming platform here allows us to develop a technique that is capable of mimicking a planner's interactive planning and decision-making process to search for a sensible solution with a commercially available system. It is useful to point out that, in this autopilot process, the prior knowledge derived from previous treatment plans are employed in every iteration of the calculation, which is different to merely use prior plans to check whether a clinical plan is consistent with previous plans\cite{li2015,moore2015,schreibmann2014}. 

\section{Conclusion}

Inverse planning in modern radiation therapy involves multiple steps of manual operation and is known to be a time-consuming process. In this work, we have developed an autonomous treatment planning strategy in C$\sharp$ programming environment to facilitate the inverse planning process. The research here is directly translatable to clinical practice as the backbone of the plan optimization in the system is built upon a commercial TPS. Additionally, the approach is quite general and allows us to incorporate empirical judgment and population-based prior knowledge into the plan selection process.  The utility of the approach has been demonstrated successfully by using a clinical prostate and a HN case. Finally, we note that the strategy lays a technical foundation for future development of big data-driven inverse planning for various disease sites. With the increased interest in big data in radiation oncology applications, the data analytics and decision-making method may prove to be useful to facilitate clinical workflow.  

