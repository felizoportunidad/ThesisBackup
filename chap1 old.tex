\chapter{Introduction} \label{chap1}

\section{Motivation}
About 1600 people die a day from cancer. In the U.S., one out of every four death is attributed to cancer. Over the course of a lifetime, one in three women and one in two men will develop cancer. Nearly two-thirds of all cancer patients will receive radiation therapy during their illness. In 2014, 88\% of patient treated with radiation therapy received external beam treatments from a linear accelerator. Treatment planning for external beam radiation therapy is a key part of a successful treatment. 
Demand for radiation therapy is expected to grow 10x faster than the supply between 2010 and 2020. The total number of patients receiving radiation therapy is expected to increase by 22\%, from 470,000 per year to 575,000 per year. In contrast, the number of radiation oncologist is only expected to increase by 2\% from 3943 to 4022. 
The process of planning external beam radiation therapy has many steps and takes several days to complete. The doctor and dosimetrist will work together to decide on the amount of radiation a patient needs. 

\section{Forward vs. Inverse Planning}
The difference between forward planning and inverse planning is that in forward planning, the planner manually selects the beam shape and the relative weightings of each beam shape. The computer then calculates the resultant dose distribution. The planner may adjust the beam shapes and intensities and asks the computer to recalculate the dose distribution. This process is repeated until an optimal dose distribution is obtained. Inverse planning was first proposed in 1988 by Brahme. With this process, the planner does not readjust the beam shapes and their associated intensities. Instead, after defining the orientation and energies of the beams, the planner specifies the desired dose constraints for the PTVs and the organs at risk. The computer optimization calculates the required beam intensities and shape to best meet the specified dose constraints. The beam is divided into small grid beamlets. The intensities of the beamlets are iteratively adjusted until the dose distribution conforms to the specified dose constraints.


\section{Current treatment planning system}
The objective of treatment planning is to provide the best possible dose distribution. Ideally, one would like to deliver the necessary dose to the planning target volume and spare all the normal tissues. But because of the patient anatomy, a certain amount of normal tissue irradiation is unavoidable.  

Optimization of the beam placement and weights is traditionally determined using iterative techniques. Most planners start with a well-established technique and make adjustments as needed to optimize dose distribution. The process is essentially both iterative and interactive. Treatment planning for modern techniques is best accomplished using inverse planning. The algorithms allow planners to specify the desired dose distribution and let treatment planning system (TPS) to generate a treatment plan as close to the desired dose distribution as possible. 


\section{Next generation treatment planning system}

\section{Definition of terms}
Treatment planning aims to determine a set of optimal parameters, including target volume, dose prescription, dose distribution, and treatment machine settings. An important step in planning is to find the tumor and critical structure location and extent. The target volume consists of a volume that includes the tumor and its occult spread to the surrounding tissues. Modern imaging modalities such as computed tomography (CT) and magnetic resonance imaging (MRI) assist the radiation oncologist in localizing the target volume. However, sufficient margin must be added to allow for uncertainty in the imaging as well as microscopic spread. The tumor seen through imaging is called the gross tumor volume (GTV). The volume that includes the GTV and the invisible microscopic disease is called the clinical target volume (CTV). The estimate of CTV is made by giving a suitable margin around the GTV. The margin includes both the gross and microscopic disease. The estimate of CTV is subjective and depends totally on clinical judgment. Added to the uncertainty of the CTV are the uncertainties of the volume localization. The location can change as a function of time because of variations in patient setup, motion of internal organs, and position instability. A planning target volume (PTV) includes the CTV and suitable margin to account for these additional uncertainties. PTV is the ultimate target volume and the primary focus of treatment planning. Adequate dose delivered to PTV assures adequate treatment of the entire disease-bearing volume, the CTV.

\section{Current treatment planning system}
The objective of treatment planning is to provide the best possible dose distribution. Ideally, one would like to deliver the necessary dose to the planning target volume and spare all the normal tissues. But because of the patient anatomy, a certain amount of normal tissue irradiation is unavoidable.  

Optimization of the beam placement and weights is traditionally determined using iterative techniques. Most planners start with a well-established technique and make adjustments as needed to optimize dose distribution. The process is essentially both iterative and interactive. Treatment planning for modern techniques is best accomplished using inverse planning. The algorithms allow planners to specify the desired dose distribution and let treatment planning system (TPS) to generate a treatment plan as close to the desired dose distribution as possible. 


\section{Next generation treatment planning system}
Recently, knowledge based planning has become a popular topic. Knowledge based planning involves using existing patient radiotherapy plan data and applying it to new patients. Several papers have used individual patient geometry to estimate probability of the dose volume histogram (DVH) for the new patient. The key difficulty is optimal objectives cannot be transferred from patient to patient as anatomy and geometry differ.
Another class of recent research falls into automatic treatment planning. From this perspective, automatic treatment planning attempts to tackle the problem of trial-and-error process which is time consuming and repetitive by focusing on the optimization aspect.  Papers have focused on the optimization aspect, namely, how to fine tune the weighting factors involved in the multicriterion optimization [Masoud paper] Other papers have focused on goal programming and lexicographic optimization. The key difficulty here is the dosimetric trade-offs are patient specific.
Both knowledge based planning and automated treatment planning have one similar goal, to make treatment plan a lot faster. They are not disparate problems, but rather, deeply connected. There is a difference in the two perspectives. This distinction arises from the fact that the doctor and dosimetrist work together to decide on the amount of radiation a patient needs. Knowledge based planning in the context of using existing patient plan data essentially attempts what the doctor does. On the other hand, the automatic planning in the context of multicriterion optimization attempts what the dosimetrist does. In this paper, we present a holistic approach that deals both issues. We attempt, for the first time, capture what the doctor and the dosimetrist do together in clinical treatment planning.
