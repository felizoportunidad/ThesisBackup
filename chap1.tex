\chapter{Introduction} \label{chap1}

\section{Motivation}

The goal of radiation therapy is to deliver therapeutic dosage to the tumor while sparing the surrounding critical organs. Unlike open surgery, external beam radiation therapy is non-invasive. Patients lie flat on a couch and a gantry rotates around the patient to deliver dose into the tumor volume inside the patient's body. Advances in imaging modalities such as computed tomography (CT), magnetic resonance imaging (MRI), and positron emission tomography (PET)) have enabled doctors to accurately visualize the patient's anatomy. Once the images are acquired, the tumor volume and critical organs are delineated through a process called contouring. Treatment planning determines an optimal set of parameters, including beam energy, orientation, and intensities to be used during treatment. Treatment planning, performed by a team of radiation oncologist, physicist, and dosimetrist, uses clinical softwares which can translate the result directly onto a treatment \mathbf{?}achine. The treatment machine has a high energy X-ray source generated by a linear accelerator (LINAC). The beams are delivered through openings of the multi-leaf collimator (MLC), which are mechanically adjustable tungsten blocks. The radiation varies at different angles along the rotation arc, which is perpendicular to the couch the patient lies on. A typical treatment takes 2 months, 5x per week, about one hour each day. Although much has been studied from the  optimization perspective, more efforts are still necessary to make the planning process faster and more efficient.   

\section{Gap between research and clinical planning}
One of the main problems in treatment planning arises from the gap between the research and clinic. Much research is performed in simplified environment (i.e. Matlab) that do not take into account the full complexity of the underlying system. On the other hand, the clinic requires results obtained from clinically acceptable softwares. Thus, research results in simplified environment are not directly translatable into the clinic. Software packages have been developed to address this gap, for example, Computational Environment for Radiotherapy Research (CERR). However, these softwares do not address the real issues encountered in the clinic. The real problems encountered in the clinic is how to satisfy the most number of patients with the least amount of effort. Problems arise because resource is limited. The number of patients far exceed the number of radiation oncologists. 