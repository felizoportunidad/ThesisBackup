\chapter{Application Programming in C\# Environment with Recorded User-Software Interactions and its Application in Autopilot of VMAT/IMRT Treatment Planning}  \label{cha:chap3}
    
       
\section{Introduction}
A clinical treatment planning software is a closed system and a task in treatment planning is realized through a series of operations (mouse clicks and/or keystrokes) within the software platform.  While automation of many clinical tasks, such as generation of a customized treatment plan report for documentation purpose or production of a clinically sensible VMAT/IMRT treatment plan, is highly desirable, its realization is hindered by two major problems. First, there is a lack of means to ``concatenate'' the manual operations of the GUI of a commercial TPS. Some vendors provide application program interface (API) toolkit, which is a set of routines, protocols, and tools for building software (e.g., in the form of a scripting), and this can be employed for certain applications \cite{xhaferllari2013, wu2005}. But the availability and applicability of the API depend heavily on the vendor. On a more fundamental level, a programming environment that allows us to interact with the TPS, such as receiving data from the TPS, assessing the received data, and providing feedback to the TPS, is not generally available.  

The purpose of this work is two-fold. First, we investigate a technique to probe the GUI and data structure of a commercial TPS (or other clinical software system) in an independent C\# programming environment. By doing so, we can interact with TPS continuously to extract the updated information of treatment planning. The need for interacting with the GUI of an application software or Web application is quite general and there are software tools developed for the Windows and UNIX/LINUX systems. Using these tools, for example, one can develop a test method to click a hyperlink in a Web application, type a value in a text box, or branch off and take different testing actions based on a value in a field. In computer science, automated tests that drive the application through its UI to examine the functionality of UI controls are known as Coded UI Tests and are widely employed to verify that the whole application, including its user interface, is functioning correctly. Coded UI Tests are particularly useful where there is validation or other logic in the user interface and are also frequently used to automate an existing manual test. 

The second purpose of this work is to present an effective strategy to autopilot the VMAT/IMRT planning process by utilizing the functionalities of a commercial TPS and the capabilities provided by the Coded UI. By recording the mouse clicks/keystrokes for specific tasks during planning as executable subroutines, we program applications in C\#, in which the recorded actions are called back to accomplish a designated task without the planner's intervention. Clinically, treatment planning is largely a trial-and-error process and relies heavily on user-software interactions. While it is desirable to automate the planning process through the development of intelligent optimization algorithm \cite{xing1999a, xing1999b, lee2013, xhaferllari2013, yang2004, cotrutz2001, yu2000, breedveld2012, tol2015}, most systems are far from being ideal and multiple iterative interactions are necessary to obtain a clinically acceptable plan.  There are multiple tradeoffs in plan selection as the inverse planning algorithm of a TPS generally contains a number of model parameters, such as the weighting factors for the involved structures. The influence of these parameters on the resultant dose distribution is not known until an optimization calculation is done, necessitating a manual trial-and-error determination of the final solution.  Instead of attempting to improve the optimization algorithm, which is typically out of the control of a TPS user, the programming platform here allows us to develop a technique that is capable of mimicking a planner's interactive planning and decision-making process to search for a sensible solution with a commercially available system. The utility of the approach in facilitating VMAT/IMRT plan optimization is demonstrated by using previously treated clinical cases.  
        
\section{Materials and Methods}
\subsection{Recording a planner's action during planning as a subroutine for subsequent applications} 
Microsoft Visual Studio Coded UI is employed to record the operations of a TPS user\cite{website2015, johnson2015}. Coded UI is targeted at providing UI accessibility and facilitating the automation of GUI manipulation. It provides a unique framework for application programming in C\# or other programming environment (Coded UI property providers support both Win32 and NET programs), and allows us to probe, identify, and manipulate UI elements of another application such as TPS. The application programming environment allows automation of the UI functionality testing and manipulations, and beyond this, definitions and implementations of functionalities that are independent of their respective implementations. Specific to medical physics applications, one of the important implications of the approach is that it enables us to access existing clinical software tools through a computer program and facilitate investigation of new ideas. This type of "plug-in" software is also useful to streamline the execution of a series of tasks for improved workflow. The strategy may also assist otherwise distinct applications with sharing data, which can help to integrate and enhance the functionalities of the applications.

A commercial Eclipse TPS (Varian Medical Systems, Palo Alto, CA) is used in this study.  We first construct a library that includes specifications for routines, data structures, object classes, and variables.  Each of the commonly used human-software interactions of mouse clicks and/or keyboard stroke (KS) is recorded using Visual Studio Coded UI automation engine and saved as action subroutine into the library. To be specific, in Table 1 we list some key examples useful for subsequent autopiloted planning.

\begin{table}
\caption{A list of some useful recording and playback actions for autopiloted planning}
\label{table:table2.1}
  \begin{tabular}{ | p{1.7in} | p{2in}  | p{2in} | }  
    \hline
    Function & Eclipse Operation \& Action in Coded UI & Application in C\# \\ \hline
    3D dose calculation & F5 & Record the ``F5'' and playback the code when dose distribution needs to be updated \\ \hline
    Open an optimization window& F10 & Record the ``F10'' and playback the code to change the Eclipse optimization parameters \\ \hline
    Adjust weighting factors or other parameters & Manually entering values into the textbox after clicking ``F10'' & Record the action of moving the mouse indicator to the textbox and entering new values. Playback with new input values. \\ \hline
    Start an optimization & Click ``Optimize'' button after clicking ``F10'' & Record the action of clicking ``optimize''. Playback when Eclipse optimizer needs to be executed with updated optimization parameters \\ \hline
    Export DVH to a file & Pull down the textbox ``show DVH view'' and select ``export DVH in tabular format'' & Record the DVH export process. Call the subroutine when updated Eclipse DVHs of the involved structures are needed \\ \hline
    Create a Boolean structure of two structures & Go to ``Contouring'' section , identify the structures, and perform a  Boolean operation & Record the process (including ``and'', ``or'', ``xor'', ``sub'', and ``not'').  Call the subroutine when a Boolean structure needs to be created \\ \hline
    Convert an isodose region into a structure & Right clicking ``dose'' on the left side menu and select ``convert isodose level to structure'' & Record the process. Playback in C\# programming environment to iteratively refine an Eclipse-optimized plan \\ \hline
    Refresh patient information & Clicking ``Save All'' and ``Reload All'' under the file pull-down menu & Record the action. Call the subroutine when existing patient information needs to be refreshed or to move the planning to next stage \\ \hline
  \end{tabular}
\end{table}

\subsection{Playing back stored action subroutines in a C\# programming environment}
A C\# programming environment is employed to utilize the recorded human-TPS interactions to facilitate VMAT/IMRT planning process.  In this environment, execution of a planning action described in Table 1 is realized by simply calling the recorded subroutine(s). A planning task is accomplished by executing a chain of pre-recorded modules, as well as syntax that analyze the intermediate results for algorithmic decision-making. To illustrate the utility of the proposed programming technique, the following presents a specific implementation of an autopiloted VMAT/IMRT planning.

\begin{figure}
	\centering
	\includegraphics[width=12cm]{./figures/chap3/fig1.pdf}
	\caption{An architectural overview of the autopiloted VMAT/IMRT plan optimization scheme implemented using the proposed technique. An outer-loop calculation (indicated by the dashed rectangular box) analyzes the generated Eclipse plan and feeds the Eclipse optimizer with updated parameters for improved dose distribution. The gray boxes are realized through the use of recorded Eclipse actions stored in a library. 
	\label{fig3.1}}
    \end{figure}
    
\subsection{Autopiloted VMAT/IMRT treatment planning}
Figure \ref{fig3.1} shows the flowchart of an autopiloted VMAT/IMRT plan optimization process with the use of recorded Eclipse actions. After PTV and other involved structures are segmented, a reference plan with similar anatomy and prescription is chosen automatically from a library to guide the search for a clinically sensible treatment plan. Here, some pre-defined geometric criteria are employed for the selection of reference plan. Specifically, the images of current case are overlaid with a candidate reference plan from a library of previously treated patients. For each structure, the signed difference of the contour points of the current and reference plans is computed \cite{schreibmann2005}. In computing the signed difference, we first introduce a polar coordinate system with its origin located at the center of mass of the PTV. Ray lines starting from the origin are introduced with an angular resolution of 2.5o. The points for a ray-line to be in and out of the structure are recorded (for the PTV, generally, the ray intercepts with its contour once). The signed difference of an intercepting point is given by subtracting the radial distance of the point in the current case from that of the reference case. A plan is not considered as a good reference if the signed difference of any intercepting point in any structure (including the skin contour) is greater than 1-3 mm, depending on the size of the structure. While the metric can certainly be further improved by introducing some heuristic weightings of the structures and points in the future, the scheme here capture the main features of the similarity assessment. In an ideal situation for two identical cases, the signed difference for all points becomes zero.  In our study, a visual inspection of the coincidence of the current and reference cases is also performed to ensure the quality of selected reference case. 

Generally speaking, the computational effort could be reduced if the information gained during the course of solving an optimization problem closely related to the current one is utilized in solving the problem at hand. In order to speed up the calculation, in this study, the beam and optimization parameters of the reference plan are used as the starting point of optimization, which is often referred to as a ``warm-start'' of optimization.  However, it is noted that a ``warm-start'' is not necessary to find the optimal plan.  Our calculation proceeds in analogous to the planning process of a human planner, with the resultant dose distribution assessed by a decision-function each time after the Eclipse optimization is done\cite{xing1999a, lee2013}. The algorithm is formally described by the following equations: 

\begin{align}\label{eq:eq3.1}
& \text{opt} \quad \text{TPS objective function with constraints} \nonumber\\
& \text{s.t.} \quad |d_{\sigma j} - d^{ref}_{\sigma j}| < \epsilon_{\sigma} \\
& \quad \quad (\sigma \in 1,2,...,N; j \in 1,2,...,J) \nonumber
\end{align}

where the optimization of TPS objective function with consideration of constraints is done on the Eclipse TPS, $d_{\sigma j}$ and $d^{ref}_{\sigma j}$ are the doses of the j-th segment of the DVH curve of structure $\sigma$ for the current and reference plans, respectively.  is the allowed deviation defined for the DVH segment of the structure  .

The autopiloted plan scheme involves the following key steps: (1) optimizing the Eclipse plan, (2) comparing the Eclipse solution with reference data, (3) deriving a new set of Eclipse optimization parameters, and (4) updating the Eclipse optimization parameters and reoptimizing the beams. The DVHs of the involved structures are exported into a .csv file upon completion of an Eclipse optimization using a pre-recorded Eclipse function (Sec. 2.1). The file is then read into the C\# program and compared with the corresponding DVHs of the reference plan. The weighting factors of Eclipse are assessed based on the difference between current and reference DVHs. For computational purpose, the DVH file is discretized with the maximal dose resolution of 0.1 cGy. The adjustments of the weightings are made toward the direction of decreasing the discrepancy between the two DVHs. Note that, while Dmin, Dmax, and Dmean are stored in the headers of the DVH file for each structure, dose volume constraints such as V95 are listed in a tabular format following the headers. A subroutine is thus written to extract the dose and volume values from the DVH file for each structure. The autopilot planning process stops if the constraints in Eq. (1) with pre-defined $\{\epsilon_{\sigma}\}$ are met (in our calculations, the value of $\epsilon_{\sigma}$ is set to be 3\% of the value in the reference plan) or if the number of iterations of the outer-loop optimization is greater than 50. To ensure that the autopiloted planning does not stop at the reference plan when there is a room for improvement, at the end of calculation, we let each resultant DVH segment to make a trial movement toward better PTV coverage or OAR sparing even if the benchmarking goal of the segment set by the reference plan is met. The trial movement is accepted if the dose to any of other DVH segments are not worsened.  By means of this last step, the autopilot process is made less dependent on the ``perfect selection'' of the reference plan.

\subsection{Case studies}
The above technique is applied to plan four clinical cases:  three VMAT head and neck cases and a fixed gantry IMRT prostate case. The selected reference plan for the first three-arc HN study is displayed in Fig. \ref{fig3.2}. This is a clinically challenging case with 50 Gy prescribed to V95 of the PTV50.  To meet the dosimetric constraints of the eyes, optic nerves and chiasm, a sagittal arc is used along with two full coplanar arcs. The photon energy for the arcs is 6 MV. The other two HN cases have 70 Gy prescribed to V95 of the PTV70 at 212 cGy/fraction. Two full coplanar arcs of 6 MV were used. For the prostate study, a typical five-field IMRT plan with similar anatomy is used as the reference (Fig. \ref{fig3.3}). In the reference plan, 6 MV photon energy is used for all beams and the beam angles are 0o, 50o, 100o, 260o, and 310o, respectively. 78 Gy is prescribed to cover V95 of the PTV in 39 fractions. For comparison, the resultant dose distributions of the autopiloting scheme are compared with the corresponding plans used for clinical treatments, which were obtained manually by a dosimetrist (signed off by a physician) and were deemed to be clinically optimal. 

  \begin{figure}
	\centering
	\includegraphics[width=12cm]{./figures/chap3/fig2.pdf}
	\caption{Isodose distributions in axial, coronal, and sagittal planes and DVHs of the involved structures of the reference for the first head and neck plan. 
	\label{fig3.2}}
    \end{figure}
    

  \begin{figure}
	\centering
	\includegraphics[width=12cm]{./figures/chap3/fig3.pdf}
	\caption{Isodose distributions in axial, coronal, and sagittal planes and DVHs of the involved structures for the reference prostate plan. 
	\label{fig3.3}}
    \end{figure}
    
\section{Results}

\subsection{HN VMAT cases}
In Fig. \ref{fig3.4}, we show the progressive improvement of the PTV and OAR doses as a function of outer-loop iteration for the first HN case. The improvement in doses of the involved structures is most prominent in the first few iterations. The dosimetric differences of the PTV and the spinal cord between the current and reference plans saturates after about 8 iterations. However, the brainstem dose continues to improve until a later stage. The DVHs of a few relevant structures at different stages of iterative calculation are displayed in Fig. \ref{fig3.5}, which provides an overall picture of the iterative optimization process of the system. 

For comparison, the VMAT plan used for the patient's treatment is presented together with the current result. A comparison of DVHs between the clinical and autopiloted planning is shown in Fig. \ref{fig3.6}. 

Fig. \ref{fig3.7} shows the DVH comparison between the clinical and autopiloted planning for the second HN case. Fig. \ref{fig3.8} show the isodose distributions og the two plans.  Similarly, Figs. \ref{fig3.9} and \ref{fig3.10} show the DVH and isodose comparisons between the clinical and autopiloted planning for the third HN case. 

It is interesting to see the minor discrepancy in the autopiloted and clinical plans. The former is obtained under the guidance of the reference solution, whereas the clinical plan was generated by a human planner independently. Clinically, it is known that inter-planner variation occurs frequently in treatment planning, especially in some sophisticated cases like the one presented here. In reality, it is important to use high quality reference plan in autopiloted planning, even though post-autopilot refinement of the treatment is feasible.

\begin{figure}
	\centering
	\includegraphics[width=12cm]{./figures/chap3/fig4.pdf}
	\caption{Change of a few dosimetric quantities as a function of outer-loop iteration. Each iteration consists of tuning optimization parameters, decision making, optimization, dose calculation, and saving parameters to file.  Panel (a) shows the percent coverage of prescription dose of PTV; Panel (b) shows the percent deviation of the maximum dose of PTV between the current and reference plans; Panel (c) shows the percent volume of the spinal cord covered by 42 Gy; Panel (d) shows the mean dose of the brainstem. 
	\label{fig3.4}}
    \end{figure}

\begin{figure}
	\centering
	\includegraphics[width=12cm]{./figures/chap3/fig5.pdf}
	\caption{DVHs of a few structures at iteration \#1 (dotted blue), \#10 (dashed red), and \#20 (solid blue).  A systematic improvement in the DVHs of the brainstem and spinal cord is observed. 
	\label{fig3.5}}
    \end{figure}

\begin{figure}
	\centering
	\includegraphics[width=12cm]{./figures/chap3/fig6.pdf}
	\caption{Final DVH of autopilot planning compared with clinical planning for the first HN VMAT case. 
	\label{fig3.6}}
    \end{figure}

\begin{figure}
	\centering
	\includegraphics[width=12cm]{./figures/chap3/fig7.pdf}
	\caption{Final DVH of autopilot planning compared with clinical planning for the second HN VMAT case. 
	\label{fig3.7}}
    \end{figure}

\begin{figure}
	\centering
	\includegraphics[width=12cm]{./figures/chap3/fig8.pdf}
	\caption{Side-by-side comparison of the isodose distributions of autopiloted (right) and clinical (left) plans for the second HN case. 
	\label{fig3.8}}
    \end{figure}

\begin{figure}
	\centering
	\includegraphics[width=12cm]{./figures/chap3/fig9.pdf}
	\caption{Final DVH of autopilot planning compared with clinical planning for the third HN case. 
	\label{fig3.9}}
    \end{figure}

\begin{figure}
	\centering
	\includegraphics[width=12cm]{./figures/chap3/fig10.pdf}
	\caption{Side-by-side comparison of the isodose distributions of autopiloted (right) and clinical (left) plans for the third HN case. 
	\label{fig3.10}}
    \end{figure}
            
\subsection{Five-field prostate IMRT}

It takes ~20 iterations for the calculation to reach the optimum. The final solution is similar to the reference plan in terms of the overall isodose distribution. It is noted that, after the system reaches to its optimal solution, a small variation (1\% to 3\%) in one or more Eclipse parameters does not cause a noticeable change in the final dose distribution and DVHs, suggesting that the final solution is stable. The final optimal isodose distribution and DVHs for the case are shown in Fig. \ref{fig3.9}. Further improvement in the OARs could not be achieved without seriously sacrificing the target dose homogeneity. The bladder DVH of the resultant plan approaches to that of the reference plan easily because of more favorable anatomy. The rectum in the case under study is geometrically closer to the PTV as compared with that of the reference case, thus its dose shows a slightly larger variation from the reference plan.

Computationally, it takes about 3 hours to complete a plan selection process, but this can be improved with more efficient programming, and, in the future, better integration with the Eclipse. We find that about 70\% of the time is spent by the Eclipse on tasks such as optimization and dose calculation, 15\% for reading from files and computation, 10\% for inputting new values, and 5\% for buffering among actions.  An algorithm that iteratively modifies the beams and objective function parameters altogether should, estimated based on the number of outer-loop/inner-loop calculation, be 15x more efficient, but this would entail tackling/modifying the objective function of the commercial TPS and becomes practically difficulty. Nevertheless, comparing with the current manual trial-and-error planning process, which could take days of a dosimetrist's time for clinically challenging cases, the proposed method eliminates the need for modifying the optimization parameters and other time-consuming tasks, such as adding Boolean structures and converting isodose curves into structure necessary to shape dose distribution toward meeting the clinical expectation. The entire calculation is done without any user invention. It thus represents a step forward in inverse planning technique. 

\begin{figure}
	\centering
	\includegraphics[width=12cm]{./figures/chap3/fig11.pdf}
	\caption{Panel (a): Final DVH of autopiloted planning as compared with manual planning for the prostate IMRT case. Panels (b)-(d): the isodose distributions of the autopiloted plan. The isodose ditribution of the plan used for this patient's treatment is almost identical to the autopilot plan and is thus not shown here. 
	\label{fig3.11}}
    \end{figure}

\section{Discussion}
In current practice, treatment planning involves testing a large number of physically realizable solutions. Intertwined interactions of TPS optimization and planner's adjustment of TPS planning parameters are needed to obtain a sensible plan. While it is desirable to automate the planning process through development of more intelligent optimization algorithm \cite{xing1999a, xing1999b, lee2013, lian2013, yu2000, hoover2015, breedveld2012}, most TPS systems are far from being ideal and multiple trial-and-errors are necessary to obtain a clinically acceptable plan.  How to make the TPS to better comprehend the planner's goal has been an active research topic since the early days of inverse planning research \cite{xing1999a, xing1999b, pugachev2002, yang2004, li2000, zhang2014, fiege2011, zhu2009, li2008, schlaefer2013, tol2015} and its perspective remains illusive. Instead of attempting to improve the optimization algorithm of the TPS, which is clearly out of the control of a user, the proposed platform and the use of an out-of-the-TPS decision-function here allows us to imitate a planner's interactive planning process to search for the optimal solution that would otherwise require much more manual effort. The approach is quite general and should be applicable to facilitate the treatment planning of other modalities such as brachytherapy \cite{cunha2010}, proton therapy\cite{liu2012, giantsoudi2013}, station parameter optimized radiation therapy (SPORT)\cite{li2013, hoover2015}, gamma knife (\cite{zhang2003}, breast planning \cite{purdie2011} or alike. 

As described in Sec. 2.3, autopilot planning entails the use of a reference plan. How to better define a set of clinically relevant geometric criteria for finding the best possible reference case(s) is an on-going research issue in knowledge based planning. Ideally, ``a priorily'' chosen reference plan is desirable to drive the optimization process toward a clinically superior plan. This can be realized in our study by selecting a reference plan that leads to positively signed difference for points entering the OARs (meaning that the ``distance of the OAR to the PTV'' in the current plan is larger than that of the reference plan). Based on our clinical experience, being able to pilot the planning process to a point that is sufficiently close to be clinically reasonable/acceptable takes up the most portion of the planning efforts and thus represents the single most challenging aspect of clinical inverse treatment planning process. In this sense, even if a reference plan that is not ``a priorily'' chosen (of course, it should be close enough to the optimal solution), the proposed approach should substantially improve the planning process. Refining a plan that is ``almost there'' requires much less effort as compared to planning from ``scratch''. In our implementation, we have made effort to make the autopiloted planning to generate a plan that may exceed the quality of the selected reference plan 

The introduction of an outer-loop decision-function based on prior clinical experience forms the basis for the autopilot algorithm to navigate through the TPS-provided solutions.  The iterative interactions of the decision-function and TPS pilot the search toward a clinically sensible solution. From the perspective of optimization, the strategy here is similar to a sequential optimization of an overall objective\cite{xing1996, zarepisheh2015}, but the objective functions for the two stages (i.e., the TPS optimization and the outer-loop determination of the TPS parameters) are different. This process is along the line of our earlier work in automated weighting factors determination \cite{yang2004, cotrutz2003, lougovski2010}. Instead of simply using prior knowledge extracted from previous clinical treatment plan(s) as a ``class solution'' or as upper/lower bounds to examine the results of the optimization calculation \cite{schreibmann2004, zhang2011, good2013, moore2015, wu2008, liu2011, wu2013}, in our approach, the reference information is utilized throughout the plan selection process. During the autopiloted planning process, the parameters used in the Eclipse optimizer are constantly updated through the comparison of current and prior knowledge characterized by reference plans. Generally speaking, it is a fundamental rule of estimation theory that the use of prior knowledge will lead to a more accurate estimation. An inclusion of even partial information could lead to more effective search of the solution space and eliminate some unnecessary uncertainties in the estimation process \cite{pugachev2002}. In image analysis and many other fields of science and engineering, it has proven extremely useful to include the prior knowledge of the system into the estimation process \cite{shieh2006, ashburner1997}. 

Generally, Coded UI does not rely on the absolute coordinates of the DVH data or GUI control buttons on the screen to perform the recording/playback, which is different from an alternative approach recently proposed by Tol et al (2015). Instead, it relies on the relative positions of the buttons. Therefore, controls can change their physical locations (layout) as long as their relative locations are maintained. If the GUI is modified in a version change of TPS to the point where certain controls are interchanged, or deleted or new controls are added, then even a manual planner would need to be trained again. In this scenario, we would need to re-record some of the actions. Recording a new action can easily be added to the preexisting library, since the code is automatically generated in Coded UI. Consistent naming of structures with approach consistent with the emerging ontology convention is important. We note that, while Coded UI is designed for the Windows environment, the principles and strategies proposed here are quite broad. The research experience gained in this platform can be translated to other platforms by other researchers to improve the manual trial-and-error process of inverse planning. There are analogous software tools for other operating systems. For Unix and Linux environment, other record-and-playback services (UNIX Session Recorder, Sikuli, Linux Desktop Testing Project) can be employed(\cite{websiteb2015}. In practice, Windows platform is employed by a large number of TPS and other clinical application software. Thus the presented strategy of using Coded UI to develop plug-in applications can be applied directly to facilitate clinical tasks or workflow that are based on the use of Windows system. Finally, in our approach the C\# program acts like a human planner, iteratively interacting with the TPS. Thus, similar to that a computer used by a manual planner is not expected to concurrently run other tasks, the TPS computer cannot be used for other purposes while the C\# program is running.

In our implementation, the data (such as the DVHs data) exchange between the inner- and outer-loop optimizers are realized mainly through export/import. In addition to these options, assertion statements can be used to extract the state of the Eclipse system. The information updates us in real-time about which operations are complete, which still on-going, and which are running into an error. It also helps to capture the cause and effect of different actions before moving the calculation to the next step. 

The reported method may have useful implications to medical physics research and clinical practice. Currently, the development of treatment planning algorithm(s) in research community is often done in simplified software platforms(\cite{deasy2003, tewell2004} with ignorance of some important geometric and physical factors \cite{deasy2003, kim2015}, which makes it difficult to experimentally validate the results using the clinical LINACs, to compare the results with current practice, and to translate the research into clinical practice. The gap between research and clinically used TPS has increased dramatically over the years, and the TPS becomes essentially a black box to the users. Researchers are thus handicapped by the lack of a clinically realistic platform for algorithmic development and for testing new ideas. The approach here enables researchers to leverage the sophisticated software subroutines existing already in a commercial TPS. The described technique enables us to utilize various software subroutines/functions in a clinical grade TPS without going into the details of their implementations, which may take years of professional engineers' efforts to develop and validate.  With the proposed approach, researchers can focus their efforts in testing their new ideas instead of spending a huge amount of efforts to ``redevelop'' the software subroutines/functions that already exist in a commercial TPS. By avoiding ``reinventing the wheels'' or repeating some well-known tasks, such as dose calculation and image registration, the researchers can focus their efforts on high-level research issues. The programming environment described here is highly interactive, which makes it easy when it comes to principle-testing and prototyping. Another important advantage of the technique here is that it may facilitate the translation of research to clinical practice because of the elimination of intermediate layers/issues as mentioned above. In terms of applications, the implemented two-loop optimization represents only one of many possible applications of the proposed strategy. Coded UI is designed to interact with the user interface in the Windows environment, thus the approach is applicable to streamline other clinical tasks that require a series of operator-software interactions in TPS or other software tools. 

\section{Conclusion}

We have demonstrated the use of recorded user-software interactions for a commercial TPS in autopilot of VMAT/IMRT treatment planning. The approach makes it easy to utilize the software tools of a clinical TPS for development of new applications and presents an uncharted area for research and applications. The strategy allows programmatically controlled execution of tasks that require a series of commands and should thus improve the clinical workflow. An autopilot optimization algorithm with incorporation of prior knowledge is implemented in combination with Eclipse TPS.  The algorithm presents a practical way to mimic the decision-making process of a planner and to pilot the plan optimization toward the reference plan, thus reducing the need for trial-and-errors in treatment planning. The autopilot method promises to simplify the clinical treatment planning. Finally, the approach should be extendable to the automation of other tasks in using a software tool. 